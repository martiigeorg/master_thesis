%%%%%%%%%%%%%%%%%%%%%%%%%%%%%%%%%%%%%%%%%%%%%%%%%%%%%%%%%%%%%%%%%%%%%%%%%%%%%%
%
% Main content starts here
%
%%%%%%%%%%%%%%%%%%%%%%%%%%%%%%%%%%%%%%%%%%%%%%%%%%%%%%%%%%%%%%%%%%%%%%%%%%%%%%

\pagenumbering{arabic}
\chapter{Introduction}
\label{sec:introduction}



\section{Problem statement}
\section{Thesis organization and contributions}

The thesis is organized as follows:
\begin{itemize}

\item[] \emph{Chapter 2:} 
\\We further continue with the theoretical part, such as .


\item[] \emph{Chapter 3:}
\\This chapter gives an overview of the state of the art of research in 


\item[] \emph{Chapter 4:} 
\\In this chapter, we are moving into the main part of the thesis - the implementation of the described approach. Additionally, the chapter presents the results of the experiments that were made. Furthermore, we evaluate and discuss the results of the experimental part of the approach.

\item[] \emph{Chapter 5:} 
\\This chapter is devoted to applications  We also discuss possible future work and collaboration with another university group of students.

\item[] \emph{Chapter 6:} 
\\We summarize the finding of  this thesis in this chapter, review the results and discuss possible tasks for future work.

\end{itemize}


In this thesis, the following contributions are made:
\begin{enumerate}

\item We propose a similarity search approach 

\item  W....

\item  We analyze the results and performance from the experimental part of the ....

\end{enumerate}







% First Paragraph
% CORE MESSAGE OF THIS PARAGRAPH:
\todo{P1.1. What is the large scope of the problem?}
\todo{P1.2. What is the specific problem?}

% Second Paragraph
% CORE MESSAGE OF THIS PARAGRAPH:
\todo{P2.1. The second paragraph should be about what have others been doing}
\todo{P2.2. Why is the problem important? Why was this work carried out?}

% Third Paragraph
% CORE MESSAGE OF THIS PARAGRAPH:
\todo{P3.1. What have you done?}
\todo{P3.2. What is new about your work?}

% Fourth paragraph
% CORE MESSAGE OF THIS PARAGRAPH:
\todo{P4.1. What did you find out? What are the concrete results?}
\todo{P4.2. What are the implications? What does this mean for the bigger picture?}

LaTeX hints are provided in \autoref{chap:latexhints}.

\chapter{Literature Review}
In this chapter, we present the theoretical background of the basic concepts of the thesis. Moreover, we define and explain different ....


\section{Quality}
The quality is defined by several International organisations.

German Industry Standard DIN 55350 Part 11 defines, that the quality “ … comprises all characteristics and significant features of a product or an activity which relate to the satisfying of given requirements” \cite{fitzpatrick1996software}.

Another definition of quality is presented in ANSI Standard (ANSI/ASQC A3/1978), describing that the "Quality is the totality of features and characteristics of a product or a service that bears on its ability to satisfy the given needs"  \citet{fitzpatrick1996software}.

ANOTHER DEFITION comes later....

\section{Quality models}

\subsection{Quality model ISO 9126-1}
text text
e

de
de
de

\begin{table}[h!]
    \caption{Sub-characteristics of the ISO 9126-1 quality modeL.}
	\begin{tabularx}{\textwidth}{X | X }
		%\hline
	    \textbf{Quality Characteristics} & \textbf{Sub-characteristics}	
	    \\ \hline
		\\ %\hline
		Functionality
        & Suitability
        
        Accuracy
        
        Interoperability 
        
        Security
        
        Compliance			 
		
		\\ \hline
	     \\ %\hline
		Reliability    
        
        &  Maturity 
        
        Fault tolerance 
        
        Recoverability 
        
        Compliance
		
		\\ \hline
	     \\ %\hline
	    Usability       
        
        & Understandability 
        
        Learnability 
        
        Operability 
        
        Compliance
    
	    \\ \hline
	  \\ %\hline
	   Efficiency &  Time behavior 
        
        Resource behavior 
        
        Compliance    
        \\ %\hline
	    \\ \hline
	    \\ %\hline
	    
	   Maintainability	
       
       & Analyzability 
        
        Changeability 
        
        Stability 
        
        Testability 
        
        Compliance
        
	    \\ \hline
	    \\ %\hline
	    
	   Portability	
       
       & Adaptability
       
       Installability 
       
       Co-existence
       
       Replaceability 
       
       Compliance
        	   
	\end{tabularx}
	\label{tab:table1}
	
\end{table}


\subsection{Quality model ISO 25010 }


textextex

\begin{table}[h!]
    \caption{Sub-characteristics of the Quality model-ISO/IEC 25010}
	\begin{tabularx}{\textwidth}{X | X }
		%\hline
	    \textbf{Quality Characteristics} & \textbf{Sub-characteristics}	
	    \\ \hline
		\\ %\hline
		Functional Suitability
       
        & Functional completeness
        
        Functional correctness
        
        Functional appropriateness		
        
        \\ %\hline
		\\ \hline
	     \\ %\hline
		Reliability    
        
        &  Maturity 
        
        Availability
        
        Fault tolerance
        
        Recoverability
        
		\\ \hline
	     \\ %\hline
	    Performance efficiency       
        
        & Time behaviour
        
        Resource utilization
        
        Capacity
        
        
	    \\ \hline
	  \\ %\hline
	   Compatibility  
       
       &  Co-existence
       
       Interoperability   
       
     
	    \\ \hline
	    \\ %\hline
	    
	   Usability	
       
       & Appropriateness recognizability
       
       Learnability
       
       Operability
       
       User error protection
       
       User interface aesthetics

       Accessibility
       
      
	    \\ \hline
	    \\ %\hline
	    
	   Security	
       
       & Confidentiality
       
       Integrity
       
       Non-repudiation
       
       Accountability
       
       Authenticity
              
       
	    \\ \hline
	    \\ %\hline
	    
	   Maintainability	
       
       &  Modularity
       
       Reusability
       
       Analysability
       
       Modifiability
       
       Testability
     
	    \\ \hline
	    \\ %\hline
	    
	   Portability	
       
       & Adaptability
       
       Installability
       
       Replaceability
        	   
	\end{tabularx}
	\label{tab:table1}
	
\end{table}

\section{Cognitive Load Theory (CLT)}
- Relevance to Code Comprehension

\section{Theory of Program Comprehension}
- Top-Down Model: Experienced programmers use existing knowledge to understand the structure and intent of code quickly.

- Bottom-Up Model: Beginners read line-by-line, building their understanding incrementally.

- Integrated Model: A mix of both, depending on experience level and task complexity


\begin{figure} [h!]
  \centering
  \includegraphics[width=\textwidth]{figures/program_Comprehension_Process.png}
  \caption{The Program Comprehension Process}
  \label{fig:AnhangsChor}
\end{figure}

\section{Eye Tracking}

•	Definition 


•	Background


•	Metrics


•	Why it’s relevant: Eye-tracking studies assume that where a person looks is what they are thinking about.


•	Key ideas:


o	Fixations (longer gaze duration) suggest higher cognitive processing effort.


o	Saccades (quick eye movements) indicate searching behavior (e.g., students scanning comments before reading the code).


o	Areas of Interest (AOI) refer to parts of the stimulus on which eye-tracking metrics are recorded. Examples could be chunks in the code editor, the requirements area of the IDE, or the console output. The AOI is usually defined by the researcher.


•	How to apply it:


o	Compare eye-tracking heatmaps to see if students focus more on comments before reading the code.


o	Measure how indentation influences reading patterns—does better indentation make scanning for logic structures easier?


\section{Studies using Eye Tracking}

Several studies analyze programming comprehension, using eye-tracking technology. This chapter reviews key studies, categorizing them based on their focus areas and summarizing their findings.

\subsubsection{Eye-Tracking Methodologies in Software Engineering}
The paper \citet{obaidellah2018survey} investigates the use of eye-tracking technology in computer programming research. A total of 63 relevant studies published between 1990 and June 2017 are analysed \citet{obaidellah2018survey}. The results indicate that eye-tracking technology has been applied to understand programmers' gaze pattern and cognitive processes. The studies reviewed in this paper focus on five main research fields: program comprehension and code understanding, non-code comprehension studies, debugging, requirements traceability, and collaborative programming \citet{obaidellah2018survey}.  The most frequently studied programming language is Java.  
For eye-tracking system, Tobii is the most commonly used.  Fixation duration and saccades are identified as primary metrics for analysing gaze patterns, and heatmaps are created for visual representation of the results.


\subsubsection{Variable Naming Conventions and Code Readability}

The study \citet{sharif2010eye}investigates whether different variable naming conventions (camel-case and underscore) impact code comprehension and readability.
The task of the participants was to recognize different variable names, while their eye movements were tracked and analysed from the researches. 
The findings indicate no significant difference in accuracy between the camel-case and underscore naming conventions \citet{sharif2010eye}. However, underscore identifiers were recognized 20\% faster compared to camel-case identifiers \citet{sharif2010eye}. Moreover, the data from the eye-tracking technology showed that camel-case identifiers led to longer gaze durations, more fixations, and a   higher cognitive load. Furthermore, novice programmers benefited more from the underscore variable naming conventions.   
Overall, the results indicate that underscore variable naming conventions are easier to process visually and potentially improve code readability.   


The eye-tracking study \citet{broberg2019using}investigates how different variable naming conventions impact on code readability.  
The test subjects analyzed code snippets containing variables in Single Word (SW), Single Letter (SL), Camel Case (MWCC), and Snake Case (MWSC) \citet{broberg2019using}. While explaining the code snippets, their eye movements were tracked. Researchers measured the time participants spent analysing the code and then created heatmaps to visualize their gaze patterns. Furthermore, participants provided feedback on which naming conventions they prefer. The results present no significant differences in readability using different variable naming conventions \citet{broberg2019using}. However, multi-word conventions like Snake Case and Camel Case were easier to read than Single Letter variable naming conventions \citet{broberg2019using}. The heatmaps presented that test subjects spent more time processing single-letter variables, leading to higher cognitive effort. Moreover, most participants personally found Snake Case and Camel Case easier to understand \citet{broberg2019using}. 
This study presents the role of meaningful variable naming conventions in code and how they improve code readability.   

\subsubsection{Code Transformations and Their Impact on Comprehension}
The eye-tracking study \citet{silva2023evaluating} investigates how different code transformations affect novice programmers’ code comprehension \citet{silva2023evaluating}. It focuses on Atoms of Confusion, Refactorings, and ifdef Annotations.

The findings show that confusing code snippets lead to difficulties in code comprehension and in increased fixations. Figure X presents the eye-gaze patterns for an obfuscated code snippet using the Conditional Expression atom, while Figure Y illustrates the clarified version of the same snippet \citet{silva2023evaluating}.
 
figure 1,2
\begin{figure} [H]
  \centering
  \includegraphics[scale=0.6]{figures/a.png}
  \caption{Obfuscated code snippet \citet{silva2023evaluating}}
  \label{fig:AnhangsChor}
\end{figure}

\begin{figure} [H]
  \centering
  \includegraphics[scale=0.6]{figures/b.png}
  \caption{Clarified version \citet{silva2023evaluating}}
  \label{fig:AnhangsChor}
\end{figure}


\subsubsection{Program Comprehension: Iteration vs. Recursion} 
The study \citet{aroobaunderstanding} investigates how students perceive recursive and iterative programs and analyzes their cognitive processes and visual attention \citet{aroobaunderstanding}. Moreover, the study inspected whether students followed a top-down approach or bottom-up approach.
The results present that there was no significant difference between recursive and iterative programs in terms of code comprehension \citet{aroobaunderstanding}.  However, visual attention differed between iteration and recursion. 

\subsubsection{Cognitive Load and Code Complexity} 

The eye-tracking paper \citet{abbad2022estimating} investigates how to identify difficult areas of code for programmers.  Participants were given different Java code snippets and mentioned the areas of the code they found difficult \citet{abbad2022estimating}. Researchers compared the eye-tracking data with the parts marked as complicated by the participants. With the collected data, they trained machine learning models to forecast which parts of the code would challenge code comprehension \citet{abbad2022estimating}. The researches achieved more than 86\% accuracy in predicting complicated parts of code snippets. The study highlights its future usage for developing tools that can enhance code review and adaptive e-learning. 



\subsubsection{Code Structure and Indentation} 


The eye-tracking study \citet{bauer2017indentations} investigates how indentation affects comprehension. Participants analyzed Java code snippets with different levels of indentations: 0, 2, 4, and 8-space. The results presented no significant relationship between comprehension and indentation \citet{bauer2017indentations}. However, the most correct answers were given when the participants analyzed code with 4-space indentation . Furthermore, the results indicated that 8-space indentation was rated as the easiest to read level of indentation, but did not improve the accuracy of the correct answers \citet{bauer2017indentations}. Overall, eye-tracking data showed no significant differences in gaze patterns across different indentation levels, and the study highlights the need for further investigation. 



Another study \citet{yorimoto2024quantitative} investigates how indentation affects code comprehension. Students were given code tasks with different levels of indentation: 0-character, 4-character, and random indentation \citet{yorimoto2024quantitative}. The researchers measured the eye movements and cognitive load of the participants. The results showed increased confusion and cognitive load when students analyzed code with unstructured indentation \citet{yorimoto2024quantitative}. Furthermore, 4-character indentation resulted in more precise but longer focus on code areas. The study mentions that indentation had little impact on small code tasks and highlights that further investigation is needed \citet{yorimoto2024quantitative}.   
\subsubsection{Code comments} 

This study \citet{bakhuizen2019comments}. investigates how incorrect code comments affect programmers' code comprehension. Participants analyzed code containing missing, incorrect, or correct comments.
The researchers used eye-tracking technology to measure how often participants read code comments. The results indicate that participants read code comments as frequently as the code \citet{bakhuizen2019comments}. Furthermore, incorrect comments can negatively affect code comprehension, leading to misinterpretations of the code. Moreover, accurate and correct comments are useful when they provide correct information for unknown functions. 

The reviewed key studies present that eye-tracking technologies effectively investigate code comprehension.  Eye-tracking data provide insights into how expert and novice programmers perceive different code structures and styles. There has been made significant progress in investigating the comprehension and readability of code by programmers.  However, there is a lack of deeper researcher on how indentation and comments influence the comprehension. 
This thesis analyzes how indentation and comments impact computer science students’ perception of code comprehension.  


\chapter{Study Design}



\section{Methodology}

\subsection{Study Approach}
The study consists of two formats: an eye-tracking study (offline study) and an online study without eye-tracking. 



\begin{center}
\begin{tikzpicture}[node distance=1cm and 3cm, align=center, font=\footnotesize]

% Define the styles for the boxes
\tikzstyle{box} = [draw, minimum height=3cm, minimum width=4cm, text width=3.5cm, inner sep=5pt]

% Create the nodes for Eye-Tracking and Questionnaire
\node (eyeTracking) [box] {Eye-Tracking \\[5mm] $\rightarrow$ Measures attention (visual gaze) \\ $\rightarrow$ Identifies cognitive load};
\node (questionnaire) [box, right=of eyeTracking] {Questionnaire \\[5mm] $\rightarrow$ Self-reports understanding \\ $\rightarrow$ Subjective measure of difficulty};

% Create the middle box for "The Study"
\node (study) [box, below=2cm of $(eyeTracking)!0.5!(questionnaire)$, text width=7cm, align=center] {The Study \\[5mm] $\rightarrow$ Both give a complete view:};

% Draw lines from Eye-Tracking and Questionnaire to "The Study"
\draw[->] (eyeTracking) -- (study);
\draw[->] (questionnaire) -- (study);

\end{tikzpicture}
\end{center}

textetx
 




\subsection{Code Snippet Selection}

The programming language used for the code snippets is Java, since it is a widely used language.  Additionally, most students, which participated in the study, are from LMU University, where Java is a key part of many computer science courses. The following code snippets formats are used in the study:



\begin{enumerate}
    \item \textbf{Comment formats:}
    \begin{itemize}
        \item \textbf{No comments:} Code snippet without any comments.
        \item \textbf{Minimal comments:} Code snippet with minimal and essential comments.
        \item \textbf{Helpful comments:} Code snippet with meaningful comments.
        \item \textbf{Redundant comments:} Code snippet with unnecessary comments.
    \end{itemize}
    
    \item \textbf{Levels of indentations:}
    \begin{itemize}
        \item \textbf{0-spaces indentation (no indentation):} Code snippet without indentation.
        \item \textbf{4 spaces indentation:} Code snippet with 4-spaces indentation.
        \item \textbf{Random indentation:} Code snippet with inconsistent indentation.
    \end{itemize}
\end{enumerate}

Figure .............  presents an example for one code snippet with minimal comments . All code snippets used in the study are provided in Appendix A.


\begin{figure} [H]
  \centering
  \includegraphics[scale=0.6]{figures/code_e.png}
  \caption{Code snippet example }
  \label{fig:AnhangsChor}
\end{figure}

\subsection{Justification of code snippets selection }
The code snippets are collected from programming lectures conducted at LMU University and adapted from the computer science exercises. The main reason why the snippets were selected from lectures is that they provide widely used examples. These examples are suitable for investigating code comprehension.  
The selection of the code snippets formats is based on the findings from previous research. It is defined that indentation levels and comment styles influence code readability and comprehension.  (e.g., Section Literature review). Therefore, the snippets in this study are designed to investigate how different levels of indentations and different commenting formats affect code comprehension.  Moreover, the code snippets are simple enough, and no additional knowledge is required. Furthermore, the usage of loops and conditionals in the snippets requires different levels of cognitive processing.    


Each snippet is focused on one dimension at a time (levels of indentations or commenting formats). This enables a simple analysis and comparison across the different formats. The study has seven different code snippets, combining three levels of indentation and four types of commenting styles. For the commenting formats, four types are selected.  The role of the snippet with no comments is to investigate whether code can be understood without any additional information.  Minimal comments test whether minimal information is better than none and improves comprehension, or makes it difficult for understanding.
Helpful comments investigate whether meaningful information enhances deeper code comprehension. Redundant comments tend to explore whether unnecessary comments make the code difficult to understand, causing cognitive overload.


For the indentation, three levels were selected. 
The role of the code snippet with no indentation is to investigate how the absence of indentation influences code readability and whether it causes cognitive overload.
The 4-space indentation is defined as standard indentation in research and tends to explore whether it enhances deeper code readability.  
Random indentation tests whether using different indentation across the code makes the snippet difficult to read and causes cognitive overload.


For the commenting formats snippets, the snippets have 4-space indentations. This ensures that the focus is only on the comments. This indentation is defined as a "control group", which allows deeper comparison across the different commenting formats. 
For the indentation code snippets, the snippets consist of no comments. This ensures that the focus is only on the indentation. Here the no comment format is also defined as a "control group", which allows deeper comparison across the different indentation levels. An overview is presented in Table ....


\begin{table}[ht]
\centering
\caption{Overview of code snippet formats}
\begin{tabular}{|p{6cm}|p{6cm}|}

\hline
\multicolumn{2}{|c|}{\textbf{Commenting formats}} \\
\hline
Type & Control Variable \\
\hline
No Comments & 4-space indentation \\
Minimal Comments & 4-space indentation \\
Helpful Comments & 4-space indentation \\
Redundant Comments & 4-space indentation \\
\hline
\multicolumn{2}{|c|}{\textbf{Indentation levels}} \\
\hline
Type & Control Variable \\
\hline
No indentation & No comments \\
4-space indentation & No comments \\
Random indentation & No comments \\
\hline
\end{tabular}
\label{tab:snippet_control}
\end{table}



\subsection{Research Variables}



\section{Questionnaire Design}


\subsection{First part of Questionnaire}
The first part of the questionnaire gives general information about the participants. 
Participants provide details about their field of study, as well as information about their academic background: their programming experience level and experience in Java.

\begin{figure} [H]
  \centering
  \includegraphics[scale=0.45]{figures/allgemein.png}
  \caption{First part of Questionnaire}
  \label{fig:AnhangsChor}
\end{figure}


\subsection{Main part of Questionnaire}

In the main part of the questionnaire are shown seven different code snippets. Their order is randomized to avoid bias. Each code snippet presents a different level of indentation and a different type of comment format.
Participants answer five questions corresponding to each code snippet. These questions are related to the code snippet’s function, ratings of the code comprehension, code indentation, and code commenting.


\begin{figure} [H]
  \centering
  \includegraphics[scale=0.45]{figures/main_p.png}
  \caption{Main part of Questionnaire}
  \label{fig:AnhangsChor}
\end{figure}



\subsection{Final part of the Questionnaire}
In the last part of the questionnaire, participants give their opinions on code indentation and code comments guidelines. They describe what they take into consideration when defining poorly commented and well-commented code. Moreover, they explain which code indentation style they use.  


\begin{figure}  [H]
  \centering
  \includegraphics[scale=0.45]{figures/last_part.png}
  \caption{Final part of the Questionnaire}
  \label{fig:AnhangsChor}
\end{figure}


 
\section{Procedure}

The eye-tracking study (offline study) and the online study without eye-tracking formats are constructed similarly, but have some differences. The detailed procedure for both formats is outlined below.

\subsection{Procedure for the Eye-Tracking Study}




\begin{enumerate}
    \item \textbf{Introduction:} At the beginning, participants answered several warm-up questions: “How are you feeling today? Do you have any experience with reading source code? Do you have any experience with eye-tracking studies?”  Then, participants are informed about the purpose of the study, data collection methods, and their right to withdraw from the study at any time.

    \item \textbf{Calibration of the Eye-Tracking System:} 
    Eye-tracker calibration is essential to ensure that the eye-tracking device correctly detects participants’ eyes and eye movements. Before calibration, participants put on the eye-tracking device and adjust three cameras: one for the screen, one for the right eye, and one for the left eye. It is essential to ensure that both eyes are correctly detected. After the adjustment is performed, participants are asked to focus on several specific points on the screen. The system records their gaze. In some cases, calibration errors occurred and recalibration was required. After the calibration is completed, the eye-tracking system started recording participants’ gaze. Participants read all relevant information themselves and then they proceeded to the questionnaire.
    
    \item \textbf{First part of Questionnaire:}  
    Participants answered several questions providing demographic, background, and programming experience information. When the students filled out these questions, they proceeded to the main part of the Questionnaire - Code Comprehension Task.

    \item \textbf{Main part of Questionnaire - Code Comprehension Task:}  Each participant is shown seven different code snippets. The order of the code snippets was randomized in order to avoid bias. The participants read each code snippet, while the eye-tracking device recorded their gaze movements, saccades, fixations, and pupil dilation. After reading each code snippet, students answered five questions. These questions are related to the corresponding code snippet. Moreover, their answers and response times were recorded.

    \item \textbf{Final part of the Questionnaire:} 
    The participants answered several questions providing information about their opinions on commenting and indentation in code.

    \item \textbf{End of the study:} 
    Participants are asked to give feedback on the difficulty level of the code snippets. Furthermore, the students mentioned their opinions on code readability and code comprehension. Additionally, they shared the challenges they faced during the eye-tracking study.

\end{enumerate}


\subsection{Procedure for the Online Study without Eye-Tracking}


\begin{enumerate}
    \item \textbf{Introduction} Participants are informed about the purpose of the study, data collection methods, and their right to withdraw from the study at any time. Participants read all relevant information themselves and then they proceeded to the questionnaire.
    
    \item \textbf{First part of Questionnaire:}
    Participants answered several questions providing demographic, background, and programming experience information. When the students filled out these questions, they proceeded to the main part of the Questionnaire - Code Comprehension Task.

    \item \textbf{Main part of Questionnaire - Code Comprehension Task:} Each participant is shown seven different code snippets. The order of the code snippets was randomized in order to avoid bias. The participants read each code snippet. After reading each code snippet, students answered five questions. These questions are related to the corresponding code snippet. Since there was no eye-tracking device, only their answers and response times were recorded.

    \item \textbf{Final part of the Questionnaire:}
    The participants answered several questions providing information about their opinions on commenting and indentation in code.
\end{enumerate}






\section{Measurements}

\section{Participants}


\section{Ethical Considerations}

\begin{enumerate}
    \item \textbf{Informed Consent: }All participants receive clear information about the purpose of the study. The participants have the right to withdraw at any time.
    
    \item \textbf{Data Protection}
    \begin{itemize}
        \item Demographic information, such as gender and age, is collected.
        \item If participants choose to receive compensation from the university, they provide their university email address.
        \item The rest of the data is anonymized and is stored for a maximum of three years.
        \item Non-anonymized data is stored for a maximum of three years.
        \item All collected data is encrypted.
        \item For eye-tracking participants, video recordings and gaze data are stored.
        \item For all participants, their questionnaire responses are stored.
        \item At any time, participants can request deletion of their data.
        \item No personal data is shared with third parties.
    \end{itemize}
\end{enumerate}

\chapter{Results}

\chapter{Discussion - Interpretation of Results}

\section{Limitations} 
Both formats of the study have several limitations that may have an impact on the results.
In the eye-tracking study only .... 10..... computer science students participated.  This small number of participants limited the generalization of the findings.  Moreover, some participants moved their heads too much.  These movements caused an inaccuracy in gaze patterns and calibration errors. Furthermore, several technical issues with eye-tracking device were reported. Sometimes the changes of ambient and screen lighting led to inaccuracy of the records.
The online study without eye-tracking also has several limitations that may have an impact on the results. 
Only .....30..... computer science students participated in the study. The small number of participants limited the generalization of the results. 
Moreover, some participants did not answer all questions. Therefore, only part of their answers were recorded, which led to missing data. 
Furthermore, the online study was not conducted in a controlled environment. Participants filled out the questionnaire in different environments, which may have an influence on their answers.  
These limitations highlight that improved technical setup and future research is needed. 


\section{Improvements}

Several improvements can be considered. First, a larger number of participants would provide more clear generalization of the findings. Additionally, the use of chin supports in the eye-tracking study may minimize participants’ head movements. It is also important to ensure consistent lighting conditions while conduction the eye-tracking study. Another suggestion is to select a laptop device that is less sensitive to ambient and screen lighting. This suggestion may reduce records errors. By implementing these improvements, future research can enhance clear generalization of the findings. 

\chapter{Conclusion and Outlook}
\label{sec:conclusion}
