% !TeX root = main-english.tex
% !TeX spellcheck = en-US
% !TeX encoding = utf8
% -*- coding:utf-8 mod:LaTeX -*-

%This smart spell only works if no changes have been made to the chapter 
%using the options proposed in preambel/chapterheads.tex.
\setchapterpreamble[u]{%
  \dictum[Albert Einstein]{We cannot solve our problems with the same level of thinking that created them}
}
\chapter{LaTeX Hints}
\label{chap:latexhints}


One can write \emph{emphasized text (rendered in italics)} and \textbf{bold text}.

\section{File Encoding and Support of Umlauts}
\label{sec:firstsectioninlatexhints}
The template offers foll UTF-8 support.
All recent editors should not have issues with that.

\section{Citations}


References are set by means of \texttt{\textbackslash cite[key]}.

\begin{filecontents*}{\democodefile}
Example: \cite{WSPA} or by author input: \citet{WSPA}.
\end{filecontents*}
\PrintDemo{style=parallel}

The following sentence demonstrates
\begin{inparaenum}[1.]
  \item the capitalization of author names at the beginning of the sentence,
  \item the correct citation using author names and the reference,
  \item that the author names are a hyperlink to the bibliography and that
  \item the bibliography contains the name prefix \textit{van der} of \textit{Wil M.\,P.\ van der Aalst}.
\end{inparaenum}

\begin{filecontents*}{\democodefile}
\Citet{RVvdA2016} present a study on the effectiveness of workflow management systems.
\end{filecontents*}
\PrintDemo{style=parallel}

The following sentence demonstrates that you can overwrite the text part of the generated label using \texttt{label} in a bibliopgrahie"=entry, but the year and the uniqueness is still generated by biber.

\begin{filecontents*}{\democodefile}
The workflow engine Apache ODE \cite{ApacheODE} executes \BPEL processes reliably.
\end{filecontents*}
\PrintDemo{style=parallel}

\begin{filecontents*}{\democodefile}
Words are best enclosed using \texttt{\textbackslash qq\{..\}}, then the correct quotes are used.
\end{filecontents*}
\PrintDemo{style=parallel}

When creating the Bibtex file it is recommended to make sure that the DOI is listed.


%%%%%%%%%%%%%%%%%%%%%%%%%%%%%%%%%%%%%%%%%%%%%%%%%%%%%%%%%%%%%%%%%%%%%%%%%%%%%%



%%%%%%%%%%%%%%%%%%%%%%%%%%%%%%%%%%%%%%%%%%%%%%%%%%%%%%%%%%%%%%%%%%%%%%%%%%%%%%
\section{Pseudocode}
%%%%%%%%%%%%%%%%%%%%%%%%%%%%%%%%%%%%%%%%%%%%%%%%%%%%%%%%%%%%%%%%%%%%%%%%%%%%%%
\autoref{alg:sample} shows a sample algorithm.
\begin{Algorithmus} %Use the environment only if you want to place the algorithm similar to graphics from TeX
  \caption{Sample algorithm}
  \label{alg:sample}
  \begin{algorithmic}
\Procedure{Sample}{$a$,$v_e$}
\State $\mathsf{parentHandled} \gets (a = \mathsf{process}) \lor \mathsf{visited}(a'), (a',c,a) \in \mathsf{HR}$
\State \Comment $(a',c'a) \in \mathsf{HR}$ denotes that $a'$ is the parent of $a$
\If{$\mathsf{parentHandled}\,\land(\mathcal{L}_\mathit{in}(a)=\emptyset\,\lor\,\forall l \in \mathcal{L}_\mathit{in}(a): \mathsf{visited}(l))$}
\State $\mathsf{visited}(a) \gets \text{true}$
\State $\mathsf{writes}_\circ(a,v_e) \gets
\begin{cases}
\mathsf{joinLinks}(a,v_e) & \abs{\mathcal{L}_\mathit{in}(a)} > 0\\
\mathsf{writes}_\circ(p,v_e)
& \exists p: (p,c,a) \in \mathsf{HR}\\
(\emptyset, \emptyset, \emptyset, false) & \text{otherwise}
\end{cases}
$
\If{$a\in\mathcal{A}_\mathit{basic}$}
  \State \Call{HandleBasicActivity}{$a$,$v_e$}
\ElsIf{$a\in\mathcal{A}_\mathit{flow}$}
  \State \Call{HandleFlow}{$a$,$v_e$}
\ElsIf{$a = \mathsf{process}$} \Comment Directly handle the contained activity
  \State \Call{HandleActivity}{$a'$,$v_e$}, $(a,\bot,a') \in \mathsf{HR}$
  \State $\mathsf{writes}_\bullet(a) \gets \mathsf{writes}_\bullet(a')$
\EndIf
\ForAll{$l \in \mathcal{L}_\mathit{out}(a)$}
  \State \Call{HandleLink}{$l$,$v_e$}
\EndFor
\EndIf
\EndProcedure
  \end{algorithmic}
\end{Algorithmus}

\clearpage



%%%%%%%%%%%%%%%%%%%%%%%%%%%%%%%%%%%%%%%%%%%%%%%%%%%%%%%%%%%%%%%%%%%%%%%%%%%%%%

%%%%%%%%%%%%%%%%%%%%%%%%%%%%%%%%%%%%%%%%%%%%%%%%%%%%%%%%%%%%%%%%%%%%%%%%%%%%%%


%With MiKTeX installation from 2012-01-16 no longer necessary.
%If a section becomes longer than one page and you want to refer to a specific place in the section with \texttt{\textbackslash{}vref}, then you should use \texttt{\textbackslash{}phantomsection} then using \texttt{vref} will also display the correct page number.

%%The link location will be placed on the line below.
%%Tipp von http://en.wikibooks.org/wiki/LaTeX/Labels_and_Cross-referencing#The_hyperref_package_and_.5Cphantomsection
%\phantomsection
%\label{alabel}
%View the example for \texttt{\textbackslash{}phantomsection} in the \LaTeX{} source code.

%Here is the example: See Section \vref{hack1} and Section \vref{hack2}.
%%%%%%%%%%%%%%%%%%%%%%%%%%%%%%%%%%%%%%%%%%%%%%%%%%%%%%%%%%%%%%%%%%%%%%%%%%%%%%
%\section{Definitions}
%%%%%%%%%%%%%%%%%%%%%%%%%%%%%%%%%%%%%%%%%%%%%%%%%%%%%%%%%%%%%%%%%%%%%%%%%%%%%%%
%\begin{definition}[Title]
%  \label{def:def1}
%  Definition Text
%\end{definition}
%
%\autoref{def:def1} shows \ldots

%%%%%%%%%%%%%%%%%%%%%%%%%%%%%%%%%%%%%%%%%%%%%%%%%%%%%%%%%%%%%%%%%%%%%%%%%%%%%%

%%%%%%%%%%%%%%%%%%%%%%%%%%%%%%%%%%%%%%%%%%%%%%%%%%%%%%%%%%%%%%%%%%%%%%%%%%%%%%